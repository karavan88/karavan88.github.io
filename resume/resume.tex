\documentclass[11pt,a4paper,sans]{moderncv}        
\moderncvstyle{banking}                           
\moderncvcolor{blue}                               
\usepackage[utf8]{inputenc}
\usepackage[scale=0.75]{geometry}
\usepackage[utf8]{inputenc}
\usepackage{enumitem}
\usepackage{xurl} 
\usepackage{fontawesome5}  % Load FontAwesome 5 icons


% Personal Information
\name{Garen}{Avanesian}
\title{Data Scientist}
% \phone[mobile]{+1~(646)~388~33~72}                   
\email{karavan0788@gmail.com}                               
% \extrainfo{Date of Birth: 12 July 1988}

\social[linkedin][linkedin.com/in/karen-avanesyan/]{karen-avanesyan} % LinkedIn profile
\social[github][github.com/karavan88]{karavan88} % GitHub profile

% Add GitHub Pages link 
\social[githubpages][https://karavan88.github.io/]{\faIcon{globe}\hspace{0.2em}\href{https://karavan88.github.io/}{Portfolio: karavan88.github.io}}

% \address{New York City, USA}{}{}                         

\begin{document}
\makecvtitle
\vspace{-1cm} % Adjust the value as needed
\section{Profile}
\cvitem{}{I’m a Statistician and Data Scientist with 7+ years of experience in turning data into actionable insights for global education and skills policies. My work bridges research and practice, engineering data systems, leading flagship reports, and building reproducible pipelines that help countries track progress toward the SDGs. I’m passionate about making data accessible and policy-relevant, especially in areas where human capital development can create more equitable opportunities. }


\section{Skills}
\cvitem{Programming Languages}{R, Python}
\cvitem{Statistical Software}{SPSS, Stata}
\cvitem{Database Management}{SQL}
\cvitem{Machine Learning \& Cloud Computing}{Microsoft Azure ML, AWS, Posit Cloud, Databricks}
\cvitem{Version Control \& DevOps}{Git/GitHub, GitHub Actions, Docker}
\cvitem{Data Visualization}{ggplot2, Shiny, Plotly, Power BI, Tableau}

\section{Experience}
\noindent \textbf{Data Science Specialist}, United Nations Children's Fund (UNICEF), New York, USA \\
\textit{September 2022 — Present} \\
Working in Data \& Analytics Section, helping countries to foster child well-being through data-driven insights, harnessing the power of analytics to support and enhance global child welfare initiatives.
\begin{itemize}[noitemsep]
    \item Automated the process for generating statistical country briefs in R, utilizing Git/GitHub, to enable effective monitoring of progress on child-related Sustainable Development Goal indicators.
    \item Optimized the ETL process between statistical estimates production and the UNICEF Data Warehouse by utilizing R and Python, significantly improving data consolidation and accessibility.
    \item Co-led the technical team for the \href{https://data.unicef.org/resources/sdg-report-2023/}{\underline{UNICEF Report on Progress on Children's Wellbeing}}, taking a leading role in enhancing methodological rigor and broadening statistical analysis based on the principles of literate programming in R.
    \item Led the \href{https://data.unicef.org/resources/ictgenderdivide/}{\underline{UNICEF Report on Bridging the Digital Gender Divide}}, leveraging large-scale data and advanced statistical methods, including mixed-effect models, to advocate for equitable programs in digital skill acquisition.
    \item Standardized complex survey data through data harmonization and using version control (Git and GitHub), improving the production of child well-being estimates.
    \item Supervised a team of 3 analysts in the development of analytical tools to collect data on quality of learning amongst children.
\end{itemize}

\noindent \textbf{Statistics Consultant}, United Nations Children's Fund (UNICEF), New York, USA \\
\textit{August 2019 — June 2022} \\
Was working in the Education Data Unit of Data \& Analytics Section.
\begin{itemize}[noitemsep]
    \item Spearheaded the use of real time-data during COVID-19 emergency response for UNICEF's Data \& Analytics, using open-source and internally collected data to analyze socio-economic impact of the pandemic on child wellbeing. 
    \item Led the production of key data-driven reports on remote learning coverage, \href{https://data.unicef.org/resources/children-and-young-people-internet-access-at-home-during-covid19/}{\underline{digital connectivity}}, and \href{https://data.unicef.org/resources/one-year-of-covid-19-and-school-closures/}{\underline{school closures}}, which influenced governmental policies and UNICEF country programs. 
    \item Developed innovative statistical methodologies like the  \href{https://data.unicef.org/resources/remote-learning-readiness-index/}{\underline{'Remote Learning Readiness Index'}} and the \href{https://data.unicef.org/resources/remote-learning-reachability-factsheet/}{\underline{'Reachability Indicator'}}, that set up UNICEF internal monitoring and program work in developing countries during the COVID-19.
    \item Co-led the \href{https://data.unicef.org/resources/mics-education-analysis-for-global-learning-and-equity/}{\underline{MICS-EAGLE initiative}} to enhance the utilization of household survey data in developing countries for sector planning.
    \item Built statistical/data analytic capacity using R, Stata, and SPSS, across international offices and ministries, facilitating better data management during emergency responses.
\end{itemize}

\noindent \textbf{Statistics Consultant}, United Nations Office on Drugs and Crime (UNODC), Vienna, Austria \\
\textit{January 2019 — August 2019} \\
Was working as a Data Analyst for Data Development and Dissemination Section.
\begin{itemize}[noitemsep]
    \item Led the data analysis for the \href{https://www.unodc.org/unodc/en/data-and-analysis/global-study-on-homicide-2019.html}{\underline{UNODC Global Study on Homicide 2019}}, authoring significant sections, creating foundational analytical methods, and producing homicide estimates.
    \item Developed a \href{https://karavan88.shinyapps.io/homicide_data_shiny_app/}{\underline{Shiny R app}} for visualizing homicide data, integrating Plotly visualizations and Leaflet maps to effectively uncover and illustrate critical patterns and trends in homicide distribution.
    \item Conducted comprehensive statistical forecasting and predictive modeling of time-series data on homicide trends.
    \item Managed extensive databases using R and SQL, integrating geospatial and crime data for a holistic view of global homicide trends.
    \item Created dynamic visualizations and web applications in R and Python to make complex data more accessible, aiding global decision-making.
\end{itemize}

\noindent \textbf{Database Manager}, International Atomic Energy Agency (IAEA), Vienna, Austria \\
\textit{October 2015 — December 2018} \\
Was working as a conference clerk and consultant to maintain conference participation data for different departments of the IAEA.
\begin{itemize}[noitemsep]
    \item Enhanced the statistical reporting system for the Conference Services Section, improving data precision and operational efficiency.
    \item Contributed to optimization of the \href{https://apps.apple.com/us/app/iaea-conferences-and-meetings/id1033279470}{\underline{{IAEA Conferences and Events mobile application}}}, significantly boosting user engagement.
    \item Developed visualization and reporting tools for conference participation data, yielding insights into participant demographics and trends.
\end{itemize}



\section{Education}
\cventry{2009 -- 2012}{PhD in Economic Sociology and Demography}{Southern Federal University/South-Russian Polytech University}
{Rostov-on-Don, Russia}{}{}
\cventry{2004 -- 2009}{Master's Degree with Honors in Sociology}{Southern Federal University}{Rostov-on-Don, Russia}{}{}

\section{Languages}
\cvitem{}{English, Russian, German}

\end{document}